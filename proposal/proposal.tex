% ETH Zurich  - 3D Photography 2015
% http://www.cvg.ethz.ch/teaching/3dphoto/
% Template for project proposals

\documentclass[11pt,a4paper,oneside,onecolumn]{IEEEtran}
\usepackage{graphicx}
% Enter the project title and your project supervisor here
\newcommand{\ProjectTitle}{Deep Relative Pose Estimation for Stereo Camera}
\newcommand{\ProjectSupervisor}{Zhaopeng Cui}
\newcommand{\DateOfReport}{March 9, 2018}

% Enter the team members' names and path to their photos. Comment / uncomment related definitions if the number of members are different than 2.
% Including photographs are optional. Photos are there to help us to evaluate your group more effectively. If you wish not to include your photos, please comment the following line.
\newcommand{\PutPhotos}{}
% Please include a clear photo of each member. (use pdf or png files for Latex to embed them in the document well)
\newcommand{\memberone}{Noah Isaak}
\newcommand{\memberonepicture}{pic2.png}
\newcommand{\membertwo}{Max Huerlimann}
\newcommand{\membertwopicture}{pic2.png}
% \newcommand{\memberfour}{Member Name}
% \newcommand{\memberfourpicture}{pic1.png}
% \newcommand{\memberfive}{Member Name}
% \newcommand{\memberfivepicture}{pic2.png}


%%%% DO NOT EDIT THE PART BELOW %%%%
\title{\ProjectTitle}
\author{3D Vision Project Proposal\\Supervised by: \ProjectSupervisor\\ \DateOfReport}
\begin{document}
\maketitle
\vspace{-1.5cm}\section*{Group Members}\vspace{0.3cm}
\begin{center}\begin{minipage}{\linewidth}\begin{center}
\begin{minipage}{3 cm}\begin{center}\memberone\ifdefined\PutPhotos\\\vspace{0.2cm}\includegraphics[height=3cm]{\memberonepicture}\fi\end{center}\end{minipage}
\ifdefined\membertwo\begin{minipage}{3 cm}\begin{center}\membertwo\ifdefined\PutPhotos\\\vspace{0.2cm}\includegraphics[height=3cm]{\membertwopicture}\fi\end{center}\end{minipage}\fi
\ifdefined\memberthree\begin{minipage}{3 cm}\begin{center}\memberthree\ifdefined\PutPhotos\\\vspace{0.2cm}\includegraphics[height=3cm]{\memberthreepicture}\fi\end{center}\end{minipage}\fi
\ifdefined\memberfour\begin{minipage}{3 cm}\begin{center}\memberfour\ifdefined\PutPhotos\\\vspace{0.2cm}\includegraphics[height=3cm]{\memberfourpicture}\fi\end{center}\end{minipage}\fi
\ifdefined\memberfive\begin{minipage}{3 cm}\begin{center}\memberfive\ifdefined\PutPhotos\\\vspace{0.2cm}\includegraphics[height=3cm]{\memberfivepicture}\fi\end{center}\end{minipage}\fi
\end{center}\end{minipage}\end{center}\vspace{0.3cm}
%%%% END OF PROTECTED LINES %%%%


%%%% BEGIN WRITING THE DOCUMENT HERE %%%%

\section{Description of the project}

This project aims at designing an end-to-end neural network which can can estimate the depth and relative pose between consecutive frames of a stereo camera. It is inspired by the work done with an unsupervised network for single-view monocular cameras \cite{zhou2017unsupervised}. In \cite{zhou2017unsupervised}, the depth is directly learned from a single image, and thus quite noisy. It also lacks of generalization to unknown scenes. With a pair of stereo images, we can get more accurate depth information, which should be able to further improve the pose estimation. What's more, it has better generalization. Given the two pairs of stereo images as input, our network will contain two parts: depth estimation and pose regression. We will adopt some existing work for depth estimation, and mainly focus on the design of the pose estimation.  The depth estimation has been shown to work well with a stereo configuration \cite{ZUMID16}. For training the KITTI dataset will be used, which provides scenes captured with stereo cameras. If the time allows, there will be attempted to fit an explanatory mask as well.

\section{Work packages and timeline}

To get familiar with the recent work, at first familiarization with the state-of-the-art algorithms will take place. Then, the network will be trained through the KITTI dataset (and possibly others), preferably in an unsupervised manner. If performance is unsatisfactory, supervised learning will be experimented with. The last step will be to validate the results on available datasets. If the time allows, the topic of semantics will be tackled. An explanatory mask could be learned simultaneously to differentiate the moving objects and improve the robustness of the method. All the code will be implemented through Python on a PC. The TensorFlow framework will be used.


\section{Outcomes and Demonstration}

The main expected outcome is an improvement of depth perception with respect to the single view monocular algorithm. This could lead to a general improving of relative pose estimation and possibly semantics. This can be demonstrated on available datasets or possibly live, if there is a stereo camera available.


\vspace{1cm}
\textbf{Instructions:}

\begin{itemize}
\item The document should not exceed two pages including the references.
\item Please name the document \textbf{3DVision\_Proposal\_Surname1\_Surname2.pdf} and upload it via the moodle.
\end{itemize}



{%\singlespace
{\small
\bibliography{refs}
\bibliographystyle{plain}}}

\end{document}